\documentclass{article}

\usepackage{fullpage,amssymb,amsthm,amsmath}
\usepackage{algorithm,algorithmic}

\newcommand{\R}{\mathbb{R}}
\DeclareMathOperator*{\argmin}{argmin}
\DeclareMathOperator*{\argmax}{argmax}

\begin{document}

\section{Submodularity}

Let $S$ be a finite set. A function $f:2^S \to \R$ is called \emph{monotone} if
$f(A) \le f(B)$ for all $A,B \subseteq S$ such that $A \subset B$.  A function
$f:2^S \to \R$ is called \emph{submodular} if for any $A,B \subseteq S$ such
that $A \subseteq B$,
$$
f(A \cup \{x\}) - f(A) \ge f(B \cup \{x\}) - f(B) \; .
$$

\subsection{Submodular Maximization}

Suppose we are given a non-negative monotone submodular function $f:2^S \to
\R$. Suppose we are given $k \ge 0$ and we want to find a set $A$ of
cardinality (at most) $k$ such that $f(A)$ is maximized. We show that a simple
greedy algorithm achieves $(1-1/e) \approx 0.632$ approximation.

The algorithm is stated as Algorithm~\ref{algorithm:submodular-greedy}.
The algorithm produces a sequence of sets $A_0, A_1, A_2, \dots, A_k$
where $A_0 \subseteq A_1 \subseteq A_2 \subseteq \dots \subseteq A_k$.
$A_0$ is the empty set and $A_k$ is the final output. Also note that $|A_i| = i$
and $f(A_0) \le f(A_1) \le \dots \le f(A_k)$ due to monotonicity of $f$.

\begin{algorithm}[t]
\caption{Greedy algorithm for submodular maximization \label{algorithm:submodular-greedy}}
\begin{algorithmic}[1]
{
\REQUIRE{Nonnegative monotone submodular function $f:2^S \to \R$}
\REQUIRE{Limit on cardinality $k$ ($0 \le k \le |S|$).}
\STATE{Initialize $A_0 = \emptyset$}
\FOR{$i=1,2,\dots,k$}
\STATE{$x_i \leftarrow \argmax_{x \in S \setminus A_{i-1}} f(A_{i-1} \cup \{x\})$}
\STATE{$A_i \leftarrow A_{i-1} \cup \{x_i\}$}
\ENDFOR
\STATE{Output $A_k$}
}
\end{algorithmic}
\end{algorithm}


\end{document}
